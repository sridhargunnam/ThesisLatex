\chapter{Conclusion}
Real world content generation for VR is an emerging research problem. VR content needs capturing for 360 degree view and 3D immersive experience. Although commercial solutions exist, the dataflow is broken resulting to long latencies and high computation power. The workflow consist of camera that captures the multiple views and offload the computing to desktop or cloud for stitching the video. These solutions needs several 1000's of CPU's if not multiple GPU's consuming consuming upto 1 kilo Watt power. Cloud based solutions will have high latency making it challenging for real-time streaming. We envision that capturing and rendering near camera helps improving the end-to-end latency and reduce the power by close integration of the capture and rendering system. 

The existing 360 camera devices have portable camera rigs that capture and offload the expensive computation to cloud, or powerful desktops. This limits the scalability of stitching operation  and increases the end-to-end latency. But performing capture and generating the VR panorama on the same device is computationally expensive. 	Our work focuses on characterizing the energy and latency of end-to-end Omni-directional(OD) Camera  systems. Through the rigorous process of building prototype and evaluating the power and latency of the system, we found that by reusing and reducing the data across different stages of pipeline, we can optimize the system for power. By pipelining and parallelizing the compute in hardware, we can reduce the latency.